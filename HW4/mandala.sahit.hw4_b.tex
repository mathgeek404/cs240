%% LyX 2.0.8.1 created this file.  For more info, see http://www.lyx.org/.
%% Do not edit unless you really know what you are doing.
\documentclass[english]{article}
\usepackage[T1]{fontenc}
\usepackage[latin9]{inputenc}
\usepackage{geometry}
\geometry{verbose,tmargin=3cm,bmargin=3cm,lmargin=3cm,rmargin=3cm}
\usepackage{fancyhdr}
\pagestyle{fancy}
\setlength{\parskip}{\smallskipamount}
\setlength{\parindent}{0pt}
\usepackage{amssymb}

\makeatletter
%%%%%%%%%%%%%%%%%%%%%%%%%%%%%% User specified LaTeX commands.
\fancyhf{}               % Clear fancy header/footer
\fancyhead[L]{Sahit Mandala, 9069858745}   % My name in Left footer
\fancyhead[R]{CS240, Section 313}  % Page number in Right footer
\makeatletter
\let\ps@plain\ps@fancy   % Plain page style = fancy page style
\makeatother

\makeatother

\usepackage{babel}
\begin{document}

\title{CS240: Homework 4}


\author{Sahit Mandala}

\maketitle

\section*{Problem 1}

P(n) = $\forall$ good sequences generated from n applications of
the constructor rules upon the foundational rule, the number of 0's
in the string equals the number of 1's

Note that $P(0)$ considers only the empty sequence (e.g. the foundational
rule construction).

We shall prove $\forall n\in\mathbb{N},P(n)$ using strong structural
induction.

\textbf{Base case} $P(0)$: At iteration 0, the only good sequence
to consider is the empty sequence, which generated by the foundational
rule. Note that the number of 0's is 0 and the number of 1's is 0,
so clearly the number of 0's and 1's are equal. Hence, we have shown
the base case $P(n)$ at $n=0$.

Next, assume the strong inductive hypothesis: Given some $n\in\mathbb{N}$,
$\forall k\in\mathbb{N}$ s.t. $x\leq n$, $P(k)$. That is, every
good sequence generated by $k$ iterations of the constructor rule,
with $\forall k\in\mathbb{N}$ s.t. $x\leq n$, has the same number
of 1's and 0's. Let us define the set S of all good sequences generated
by some $m\in\mathbb{N}$ applications of the constructor rule, with
$m\leq n$. Then for all sequences s in S, s has the same number of
1's and 0's by the strong inductive hypothesis, and S contains every
good sequence generated by up to n applications of the constructor
rule. 

We want to now show that $P(n+1)$ follows. Let some good sequence
generated from $n+1$ applications of the constructor rules upon the
foundational rule be given, which we shall call $s_{n+1}$. Then $s_{n+1}$
was generated using some 2 elements in S using one of the constructor
rules. We shall consider every possible construction of $s_{n+1}$:


\subsubsection*{Case 1:}

Suppose that $s_{n+1}$ has the form $0s1t$ for some $s,t\in S$. 

We know that s,t each have the same number of 0's, 1's. So lets say
$i_{s}$ = \# of 0's in s = \# of 1's in s and $i_{t}$ = \# of 0's
in t = \# of 1's in t. 

Notice that the \# of 0's in $s_{n+1}$ = the \# of 0's in $0s1t$
= 1 + \# of 0's in $s$ + 0 + \# of 0s in t = $1+i_{s}+i_{t}$, since
the number of 0's is additive across subsequences and the sequences
0,1 clearly have 1,0 0's respectively. Furthermore, the \# of 1's
in $s_{n+1}$ = the \# of 1's in $0s1t$ = 0 + \# of 1's in $s$ +
1 + \# of 1's in t = $i_{s}+1+i_{t}$, by similar logic. But clearly,
$1+i_{s}+i_{t}=i_{s}+1+i_{t}$, so $s_{n+1}$ has the same number
of 0's and 1's


\subsubsection*{Case 2: }

Suppose that $s_{n+1}$ has the form $1s0t$ for some $s,t\in S$. 

Again, We know that s,t each have the same number of 0's, 1's. So
lets say $i_{s}$ = \# of 0's in s = \# of 1's in s and $i_{t}$ =
\# of 0's in t = \# of 1's in t. 

Notice that the \# of 0's in $s_{n+1}$ = the \# of 0's in $1s0t$
= 0 + \# of 0's in $s$ + 1 + \# of 0s in t = $i_{s}+1+i_{t}$, again
noting that the number of 0's is additive across subsequences. Furthermore,
the \# of 1's in $s_{n+1}$ = the \# of 1's in $1s0t$ = 1 + \# of
1's in $s$ + 0 + \# of 1's in t = $1+i_{s}+i_{t}$, by similar logic.
But clearly, $1+i_{s}+i_{t}=i_{s}+1+i_{t}$, so $s_{n+1}$ has the
same number of 0's and 1's.

Overall, we have shown that $s_{n+1}$ has the same number of 0's
and 1's. Because $s_{n+1}$ was any arbitrary good sequence generated
from $n+1$ applications of the constructor rules upon the foundational
rule, we have shown all good sequences generated by $n+1$ applications
of the constructor rule have the same number of 0's and 1's. Whence,
we have shown $P(n+1)$, from which it follows that $\forall n\in\mathbb{N},P(n)$
holds.


\section*{Problem 2}


\subsection*{Part A}

Let our loop invariant be: P(m) = on the $m^{th}$ iteration of the
loop on line (2), $0\leq i\leq n$ and $x\not\in A[0,(i-1)]$.

First, lets consider the case that A is empty. Then $n=0$. Then,
during the 0th iteration, $i=0$ and clearly, $x\not\in A[0,-1]=[]$.
So then $i=n$, which means the loop at (2) is skipped entirely. Hence,
the loop invariant holds (for the only iteration, 0).

Now we shall show induction on m in P(m), assuming that the array
A is nonempty.

\textbf{Base case} $P(0)$: On the 0th iteration, the loop has iterated
0 times, so $i=0$ from line (1). Since we assumed A is nonempty,
$0\leq n$, so $0\leq i\leq n$. Also, $x\not\in A[0,0-1]=A[0,-1]=[]$,
the empty subarray. Hence, we have shown $P(0)$.

\textbf{Inductive Hypothesis:} Next, suppose that $P(m)$ for some
$m\in\mathbb{N}$. That is, on the $m^{th}$ iteration, $0\leq i\leq n$
and $x\not\in A[0,(i-1)]$. We want to show $P(m+1)$. 

If the $m+1^{th}$ iteration does not occur, then we are done. 

So assume that the $m+1^{th}$ iteration occurs. Because this iteration
occurs, we know that the equality on (2) was satisfied, so $i<n$.
If the conditional on (3) was triggered, then the return statement
is triggered, so the loop does not complete. Because we are considering
the case that this iteration completes, we can infer that this conditional
is not satisfied, so $x\neq A[i]$. Since $x\not\in A[0,(i-1)]$ and
$x\not\in A[i]$, then $x\not\in A[0,i]$. Furthermore, after (4),
i is incremented to $i+1$. This completes the iteration of the loop.
Since we know $i<n$ and both are integral values, then $i+1\leq n$;
and clearly, $0\leq i+1$ since $0\leq i$. So overall, after the
$m+1^{th}$ iteration, $0\leq i+1\leq n$ and $x\not\in A[0,i]=A[0,(i+1)-1]$.
Hence, we have shown $P(m+1)$ and thus, overall, have shown that
$\forall m\in\mathbb{N},$ the loop invariant $P(m)$ holds.


\subsection*{Part B}


\subsubsection*{Partial correctness:}

We assume that the program terminates and go on to show that whenever
the program terminates, we get a correct solution. There are 2 possible
case of termination:

\textbf{Case 1}: 

Suppose the program terminates at (5). Because we return -1, we want
to show that $x\not\in A[0,n-1]$ (that is, x is not in the array
A). Reaching (5) implies that the loop conditional was not satisfied
after some $k^{th}$ iteration of the loop, with $k\in\mathbb{N}$.
By our loop invariant, we know after that iteration, $0\leq i\leq n$
and $x\not\in A[0,i-1]$. Because the loop conditional was not satisfied,
$i\geq n$ (since $i\not<n$). This implies that $i=n,$ from which
it follows that $x\not\in A[0,i-1]=A[0,n-1]$. So $x$ is in fact
not in the entire array. Whence, we have shown correctness when terminating
at line (5).

\textbf{Case 2}: Suppose that the program terminates at line (3).
That is, we return $i$. We want to show that $i$ is the smallest
index such that $x=A[i]$. Reaching (3) implies that the if conditional
was satisfied after some $k^{th}$ iteration of the loop, with $k\in\mathbb{N}$
(so we reach (3) on the $k+1^{th}$ {[}partial{]} iteration of the
loop). After this $k^{th}$iteration, we know that $0\leq i\leq n$
and $x\not\in A[0,i-1]$. Again, because the if conditional was satisfied,
we know that $x=A[i]$, which clearly shows $x\in A[0,n-1]$. Because
$x\not\in A[0,i-1]$, we know that $\forall j\in\mathbb{N}$ with
$0\leq j\leq i-1$, $x\not=A[j]$. That is, for all indices $j$ smaller
than $i$ , $A[j]\neq x$. So clearly, $i$ is the smallest index
satisfying $x=A[i]$.

Overall, we have shown that when the program terminates, we have a
correct solution in either case.


\subsubsection*{Termination:}

We know that $i=0$ and are given some $n\in\mathbb{N}$, A, x. To
prove termination, we can show that this loop always terminates in
at most n iterations of the loop. I claim that after the $m^{th}$
iteration of the loop for any $m\in\mathbb{N}$, the program terminated
or $i=m$. We shall use induction on $m$ in $P(m)$ := $(i=m)$

\textbf{Base case} $m=0$: On the 0th iteration, $i=0=m$.

\textbf{Inductive hypothesis}: Let some $k\in\mathbb{N}$ be given.
Assume $P(k)$. That is after the $k^{th}$ iteration, $i=k$ or the
program terminated.

We now want to show that $P(k+1)$ holds. That is, the value of $i$,
say $i'$, after the $k+1^{th}$ iteration satisfies $i'=k+1$

If the $k+1^{th}$ iteration does not occur (including if the $k^{th}$
iteration terminated) or the program terminates on (3) of the $k+1^{th}$
loop, we are done. 

Now suppose that the $k+1^{th}$ iteration does occur but does not
terminate on (3). Then we know $i=k$ from the $P(k)$ (after all,
the $k^{th}$ iteration did not terminate). During the $k+1^{th}$
iteration, line (3) does not change the value of $i$, so we can ignore
it. Note that we will not trigger (3) by our assumption that the $k+1$
iteration does not terminate here. Then, in line (4), $i$ is incremented
by $i$, and since $i=k$, $i'=i+1=k+1$ is the value after the $k+1^{th}$
iteration, as expected. Hence, we have shown by induction that after
the $k^{th}$ iteration, $i=k$ or the program terminated.

To utilize this lengthy proof, first note that if the loop terminates
on the $m^{th}$ iteration for some $m\leq n$ (here, we expect termination
on (3), but that doesn't explicitly matter), we are done. So consider
the case that $\forall m\in\mathbb{N}$ with $m\leq n$, the $m^{th}$
iteration does not terminate. Then, after the $n^{th}$ iteration,
we know that $i=n$ after the loop completes. So then the loop conditional,
$i=n\not<n$ is not satisfied, and the loop is completed, causing
the program to terminate on (5). Ennumerating all cases, this shows
that the program always terminates in at most $n$ (a finite number)
iterations. 
\end{document}
